%% start of file `template.tex'.
%% Copyright 2006-2015 Xavier Danaux (xdanaux@gmail.com).
%
% This work may be distributed and/or modified under the
% conditions of the LaTeX Project Public License version 1.3c,
% available at http://www.latex-project.org/lppl/.


\documentclass[11pt,a4paper,sans]{moderncv}
\moderncvstyle{classic}
\moderncvcolor{blue}

% adjust the page margins
\usepackage[scale=0.75]{geometry}
\setlength{\hintscolumnwidth}{3.6cm}

\name{David}{Eklov}
\address{290 South Castanya Way}{94028 Menlo Park, CA}
\phone[mobile]{+1~(512)~423~9232}
\email{david.eklov@gmail.com}

\begin{document}
\makecvtitle

Software architect/developer with deep experience in a wide variety of areas.
Have worked on bootloaders, operating system kernels, hardware simulators and
compilers all the way up to cloud devops infrastructure, microservice
architectures, data pipelines, statistical models and machine learning. Have
built and led local and distributed teams of various sizes in several of these
areas. I find great satisfaction in mastering new technologies and to cross
pollinate key ideas from different areas of software development to solve new
and challenging problems.

\section{Professional Experience}
\cventry{Mar, 2018--present}{Chief Scientist}{unitQ}{San Mateo, CA}{}{
Architected and led the engineering teams that built the UnitQ Monitor. The
Monitor ingests and analyzes public and proprietary user feedback, e.g. support
emails, tweets, app reviews, etc., and identifies bugs in our customers’
software services reported by their users.
\newline
\newline
\textbf{Highlights}:
\begin{itemize}
\item Built and led two distributed engineering teams
\begin{itemize}
\item Frontend and Backend - 7 engineers
\item Data Science and Machine Learning - 5 engineers
\end{itemize}
\item Architected, led and contributed to the development of
\begin{itemize}
\item Kubernetes based backend with 15+ micro services
\item Scalable data pipeline including state-of-the-art ML/NLP models
\item Highly available data crawlers for 10+ different data sources
\item Infrastructure as code and CI/CD pipeline
\item Tooling for training ML models on Google AI platform
\end{itemize}
\item Led and contributed to the development of several ML/NLP algorithms
\begin{itemize}
\item Various BERT based text classification models
\item BERT + human-in-the-loop based text clustering algorithm
\end{itemize}
\item Designed tools, defined and ran processes to manage 10+ humans in the loop
\item Helped raise \$11M series A led by Gradient Ventures (Google's AI fund)
\end{itemize}
\ \newline
\textbf{Technologies Used}: Kubernetes, Elasticsearch, DynamoDB, Aurora, Kinesis, Terraform, Google AI Platform, Tensorflow 2.0, Huggingface Transformers, scikit-learn, Scala, Python, Java.
}
\cventry{Oct, 2012 -- Mar, 2018}{Senior Performance Architect}{Samsung}{Austin, TX}{}{
\begin{itemize}
\item Designed and implemented a high-performance, full-system, functional, multicore
simulator for the ARM architecture capable of booting unmodified Linux and
Android software stacks. The simulator was written in C++. I did most of the
initial implementation work including: memory model, interrupt routing, DMA
engine, various device models such as interrupt, block and network device controllers,
and the internal threading model for enabling non-blocking I/O. I also did the Linux
and Android bring-up.
\item Worked on source-to-source compiler, written in Scala, for compiling a pseudo code
description of the ARM instruction set into optimized C++ code used to simulate
instructions in high-performance, multicore CPU model.
\item Developed and maintained user-space tools and Linux drivers for proprietary
performance monitoring units. Helped configure and tune the kernel and system software
for other teams working with Linux and Android workloads.
\end{itemize}
}

\cventry{Aug, 2007 -- Dec, 2008}{Software Engineer}{Acumem}{Uppsala, Sweden}{}{
Acumem was founded in 2006 to develop optimization tools for multithreaded programs
using statistical analysis of memory access patterns. I joined the company in an early
stage and worked on the core cache modeling technology for Acumem’s first software
release. Acumem was acquired by Rough Wave Software in 2010.
}

\cventry{Jan, 2007 -- Jun, 2007}{Intern, Software Engineer}{Vmware}{Palo Alto, CA}{}{
Proposed, implemented and evaluated a storage IO bandwidth manager for the ESX
server.
}

\cventry{Jan, 2006 -- Jun, 2006}{Intern, Quality Assurance}{Vmware}{Palo Alto, CA}{}{
Implemented a testing framework to test the operating system kernel of the ESX server.
}

\section{Education}
\cventry{Sep, 2007 -- Oct, 2012}{Ph.D.}{Uppsala University}{Uppsala, Sweden}{}{
Computer Architecture\newline
Developed novel methods for software performance analysis on multicore architectures.
The methods provide quantitative data of how applications’ utilize resources in the
memory hierarchy, such as cache capacities and bandwidths. This data has been used
to predict and explain how applications scale with increasing core counts, to determine
how the performance of applications is impacted by resource sharing and to guide
compiler optimizations.
}

\cventry{Sep, 2001 -- Dec, 2006}{M.Sc.}{Chalmers University of Technology}{Gothenburg, Sweden}{}{
Electrical Engineering and Applied Mathematics
}

\section{Software and Services}
\subsection{Programming Languages}
\cvitem{C} {\small
Firmware, kernel and system-level code
}
\cvitem{C++} {\small
CPU and SoC simulators, discrete event simulators and system-level code
}
\cvitem{Python} {\small
Machine learning models, data analysis and processing and microservices
}
\cvitem{Scala} {\small
Source code parsers, analysis and optimization passes, back-end code generation
}

\subsection{Technologies}
\cvitem{} {\small
Docker, Kubernetes, Helm, Elasticsearch, DynamoDB, Kinesis, Terraform
}

\subsection{Services}
\cvitem{} {\small
Github, CircleCI, Drone CI, Datadog, Papertrail, Pingdom, AWS, Google Cloud, Crowdflower, Groundtruth
}

\subsection{Libraries}
\cvitem{} {\small
Tensorflow 2.0/Keras, Huggingface Transformers, scikit-learn, numpy, Pandas, Flask, Cats, fs2, http4s, Doobie
}

\subsection{Operating Systems}
\cvitem{} {\small
Ubuntu, Gentoo, OSX, Android (AOSP)
}

\section{Publications}
\subsection{Refereed Conference Publications}
{\small
\cvitem{} {
    David Eklov, Nikos Nikoleris, Erik Hagersten. ``A software based profiling
    method for obtaining speedup stacks on commodity multi-cores'', In
    \emph{Proceedings of International Symposium on Performance Analysis of
    Systems and Software}, March 2014
}
\cvitem{} {
    Andreas Sembrant, David Eklov and Erik Hagersten. ``Efficient
    Software-based Online Phase Classification'', In \emph{Proceeding of
    the International Symposium on Workload Characterization}, November 2011
}
\cvitem{} {
    David Eklov, Nikos Nikoleris, David Black-Schaffer and Erik Hagersten.
    ``Cache Pirating: Measuring the Curse of the Shared Cache'', In
    \emph{Proceeding of the International Conference on Parallel Processing},
    September 2011 (\textbf{Received best paper award})
}
\cvitem{} {
    David Eklov, David Black-Schaffer and Erik Hagersten. ``Fast Modeling
    of Cache Contention in Multicore Systems'', In \emph{Proceedings of the
    International Conference on High Performance and Embedded Architecture
    and Compilation}, January 2011 (\textbf{Received best paper award})
}
\cvitem{} {
    Andreas Sandberg, David Eklov and Erik Hagersten. ``Reducing Cache
    Pollution Through Detection and Elimination of Non-Temporal Memory
    Accesses'', In \emph{Proceedings of Supercomputing}, November 2010
}
\cvitem{} {
    David Eklov and Erik Hagersten. ``StatStack: Efficient Modeling of LRU
    Caches'', In \emph{Proceedings of International Symposium on Performance
    Analysis of Systems and Software}, March 2010
}

\subsection{Book Chapter}
{\small
\cvitem{} {
    Erik Hagersten, David Eklov and David Black-Schaffer. ``Efficient Cache
    Modeling with Sparse Data''. Chapter in "Processor and System-on-Chip
    Simulation'', editors Olivier Temam and Rainer Leupes. Springer, 2010
}
}

\end{document}

